\LARGE\bf{Zdrojový kód súboru AeroShield.m}
\label{AeroShield.m}
\vspace{1cm}
\begin{lstlisting}[numbers=left,basicstyle=\scriptsize,caption={Zdrojový kód súboru AeroShield.m.},captionpos=b]	
classdef AeroShield < handle

properties
	arduino;
	as5600;
end
properties(Constant)
	AERO_UPIN = 'D5'; 
	AERO_RPIN = 'A3';
	VOLTAGE_SENSOR_PIN = 'A2';
	voltageReference = 5.0;
	ShuntRes = 0.1;
	correction1 = 4.1220;
	correction2 = 0.33;
	repeatTimes = 50;
end

methods
	function begin(AeroShieldObject) 
		AeroShieldObject.arduino = arduino();
		AeroShieldObject.as5600 = device(AeroShieldObject.arduino,'I2CAddress',0x36); 
		configurePin(AeroShieldObject.arduino,AeroShieldObject.AERO_UPIN, 'DigitalOutput') 
		disp('AeroShield initialized.') 
	end
	function startangle = calibration(AeroShieldObject)
		write(AeroShieldObject.as5600, 0x0c, 'uint8');
		write(AeroShieldObject.as5600, 0x0d, 'uint8');
		startangle = read(AeroShieldObject.as5600, 1, 'uint16');
	end
	function PWM = referenceRead(AeroShieldObject)
		PWM= readVoltage(AeroShieldObject.arduino, AeroShieldObject.AERO_RPIN);
	end
	function actuatorWrite(AeroShieldObject, PWM) 
		writePWMVoltage(AeroShieldObject.arduino, AeroShieldObject.AERO_UPIN, PWM);
	end
	function RAW = getRawAngle(AeroShieldObject)
		write(AeroShieldObject.as5600, 0x0c, 'uint8');
		write(AeroShieldObject.as5600, 0x0d, 'uint8');
		RAW = read(AeroShieldObject.as5600, 1, 'uint16');
	end
	function currentMean = getCurrent()
	  for r = 1:repeatTimes
		voltageValue = readVoltage(AeroShieldObject.arduino, AeroShieldObject.VOLTAGE_SENSOR_PIN);
		voltageValue= (voltageValue * voltageReference) / 1024;
		Current= Current + correction1-(voltageValue / (10 * ShuntRes)); 
	  end
		currentMean= Current/repeatTimes;
		currentMean= currentMean-correction2;
	  if currentMean < 0.000 
		currentMean= 0.000; 
	  end
		current= 0;  
		voltageValue=0;   
	end 
end
end
\end{lstlisting}