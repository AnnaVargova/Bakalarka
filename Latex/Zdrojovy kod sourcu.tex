\LARGE\bf{Zdrojový kód súboru AeroShield.cpp}
\vspace{1cm}
\begin{lstlisting}[caption={Zdrojový kód súboru AeroShield.cpp.},captionpos=b]	
#include "AeroShield.h"         // Include header file

// Initializes hardware pins
float AeroShield::begin(bool isDetected){                             // Board initialisation
	pinMode(AERO_UPIN,OUTPUT);  		                                  // Actuator pin
	
	#ifdef ARDUINO_ARCH_AVR                                             // For AVR architecture boards
	Wire.begin();                                                     // Use Wire object
	#elif ARDUINO_ARCH_SAM                                              // For SAM architecture boards
	Wire1.begin();                                                    // Use Wire1 object
	#elif ARDUINO_ARCH_SAMD                                             // For SAMD architecture boards
	Wire.begin();                                                     // Use Wire object
	#endif
	
	
	if(isDetected == 0 ){                                           // If magnet not detected go on
		while(1){                                                           // Go forever until magnet detected 
			if(isDetected == 1 ){                                           // If magnet detected
				AutomationShield.serialPrint("Magnet detected \n");         // Print information then break
				break;
			}
			else{                                                           // If magnet not detected 
				AutomationShield.serialPrint("Can not detect magnet \n");   // Print information then go back to check while statement
			}
		}
	}       
} 


float AeroShield::convertRawAngleToDegrees(word newAngle) {             // Function for converting raw angle(0-4096) to degrees(0-360degree) 
	float retVal = newAngle * 0.087;                                      // 360degree/4096=0.087degree times the raw value
	ang = retVal;                               
	return ang;                                                           // Return angle value in degrees 
}


float AeroShield::calibration(word RawAngle) {                          // Calibration 
	AutomationShield.serialPrint("Calibration running...\n");             // Print info 
	startangle=0;                                                         // Zero out Variable(precaution)
	analogWrite(AERO_UPIN,50);                                            // Power the actuator, swing the pendulum 
	delay(250);                                                           // Wait for 0.25s 
	analogWrite(AERO_UPIN,0);                                             // Actuator powered off, pendulum goes to zero position
	delay(4000);                                                          // Wait for pendulum to stop oscilating 
	
	startangle = RawAngle;                                                // Save the value of zero pozition in raw format 
	analogWrite(AERO_UPIN,0);                                             // Actuator powered off(precaution)
	for(int i=0;i<3;i++){                                               // Simple sound indication of successful calibration 3 beeps
		analogWrite(AERO_UPIN,1);                                         // Actuator powereded just a bit so the rotor doesn't turn just beep 
		delay(200);                                                       // wait 
		analogWrite(AERO_UPIN,0);                                         // Actuator powered off
		delay(200);                                                       // wait 
	}
	
	AutomationShield.serialPrint("Calibration done");
	return startangle;                                                  // Return start angle
}


float AeroShield::referenceRead(void) {                                                  // Reference read
	referencePercent = AutomationShield.mapFloat(analogRead(AERO_RPIN), 0.0, 1024.0, 0.0, 100.0);   // Remapps the analog value from original range 0.0-1023 to percentual range 0.0-100.0
	return referencePercent;                                                                 // Returns the percentual position of potentiometer runner
}

void AeroShield::actuatorWrite(float PotPercent) {                                         // Actuator write
	float mappedValue = AutomationShield.mapFloat(PotPercent, 0.0, 100.0, 0.0, 255.0);       // Takes the float type percentual value 0.0-100.0 and remapps it to range 0.0-255.0
	mappedValue = AutomationShield.constrainFloat(mappedValue, 0.0, 255.0);                  // Constrains the remapped value to fit the range 0.0-255.0 - safety precaution
	analogWrite(AERO_UPIN, (int)mappedValue);                                                // Write remapped value to actuator pin 
}

float AeroShield::currentMeasure(void){                                                    // Measuring current drawn by DC motor 
	for(int i=0 ; i<repeatTimes ; i++){                                                      // Function for callculating mean current value 
		voltageValue= analogRead(VOLTAGE_SENSOR_PIN);                                         // Read a value from the INA169 
		voltageValue= (voltageValue * voltageReference) / 1024;                               // Remap the ADC value into a voltage number (5V reference)
		current= current + correction1-(voltageValue / (10 * ShuntRes));                      // Equation given by the INA169 datasheet to
		// determine the current flowing through ShuntRes. RL = 10k
	}                                                                                     // Is = (Vout x 1k) / (RS x RL)
	
	float currentMean= current/repeatTimes;                                                 // Callculating mean current value 
	currentMean= currentMean-correction2;                                                   // Small correction of current value(determined by multimeter)
	if(currentMean < 0.000){                                                                // Correction for occasional bug causing the value to be negative. 
		currentMean= 0.000;                                                                  // When it so happens, zero out the value. 
	}
	
	current= 0;                                                                              // Zero out current value        
	voltageValue= 0;                                                                         // Zero out voltage value  
	return currentMean;                                                                      // Return mean current value 
	
}

word AeroShield::getRawAngle()                                                             // Function for getting raw pendulum angle data 0-4096
{
	return readTwoBytes(_raw_ang_hi, _raw_ang_lo);                                           // Another function for communication with senzor, called from this library 
}


int AeroShield::detectMagnet()                                                             // Function for detecting presence of magnet 
{
	int magStatus;                                                                           // Auxiliary variable
	int retVal = 0;                                                                          // Auxiliary variable
	magStatus = readOneByte(_stat);                                                          // Another function for communication with senzor, called from this library                         
	
	if (magStatus & 0x20)
	retVal = 1;
	
	return retVal;                                                                           // Return value 
}


int AeroShield::getMagnetStrength()                 // Function for getting the strength of magnet 
{
	int magStatus;                                    // Auxiliary variable
	int retVal = 0;                                   // Auxiliary variable
	magStatus = readOneByte(_stat);                   // Another function for communication with senzor, called from this library     
	
	if (detectMagnet() == 1)                          // Return 0 if no magnet is detected
	{
		retVal = 2;                                     // Return 2 if magnet is just right
		if (magStatus & 0x10)
		retVal = 1;                                   // Return 1 if magnet is too weak
		else if (magStatus & 0x08)
		retVal = 3;                                   // Return 3 if magnet is too strong
	}
	
	return retVal;                                    // Return value 
}


int AeroShield::readOneByte(int in_adr)             // Function for communicating with the senzor using 1 Byte 
{
	int retVal = -1;
	Wire.beginTransmission(_ams5600_Address);         // Initialise wire transmission 
	Wire.write(in_adr);                               // Write 4 bits 
	Wire.endTransmission();                           // End wire transmission 
	Wire.requestFrom(_ams5600_Address, 1);            // Request answer  
	while (Wire.available() == 0);                    // Wait for returning bits 
	
	retVal = Wire.read();                             // Store returning bits 
	
	return retVal;                                    // Return stored bits 
}


word AeroShield::readTwoBytes(int in_adr_hi, int in_adr_lo)          // Function for communicating with the senzor using 2 Bytes 
{
	word retVal = -1;
	
	/* Read Low Byte */
	Wire.beginTransmission(_ams5600_Address);        // Initialise wire transmission 
	Wire.write(in_adr_lo);                           // Write 4 bits 
	Wire.endTransmission();                          // End wire transmission 
	Wire.requestFrom(_ams5600_Address, 1);           // Request answer  
	while (Wire.available() == 0);                    // Wait for returning bits 
	
	int low = Wire.read();                           // Store first returning bits 
	
	/* Read High Byte */
	Wire.beginTransmission(_ams5600_Address);        // Initialise wire transmission 
	Wire.write(in_adr_hi);                           // Write 4 bits   
	Wire.endTransmission();                          // End wire transmission 
	Wire.requestFrom(_ams5600_Address, 1);           // Request answer  
	while (Wire.available() == 0);                    // Wait for returning bits 
	
	word high = Wire.read();                         // Store second returning bits 
	
	high = high << 8;                                // bitwise left shift
	retVal = high | low;
	
	return retVal;                                   // Return stored bits 
}	
\end{lstlisting}