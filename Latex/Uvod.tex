\chapter*{Úvod}
\label{UVOD}
\addcontentsline{toc}{chapter}{Úvod}

%V úvode autor podrobnejšie ako v predhovore, pritom výstižne a krátko charakterizuje stav poznania alebo praxe v špecifickej oblasti, %ktorá je predmetom záverečnej práce. Autor presnejšie ako v predhovore vysvetlí ciele práce, jej zameranie, použité metódy a stručne %objasní vzťah práce k iným prácam podobného zamerania. V úvode netreba zachádzať hlbšie do teórie. Netreba podrobne opisovať metódy, %experimentálne výsledky, ani opakovať závery prípadne odporúčania.
%Úvod začína na novej  strane.

Cieľom tejto bakalárskej práce je návrh, výroba a naprogramovanie modernej učebnej pomôcky AeroShieldu (ďalej len „shield”), ktorý slúži na výuku základov teórie riadenia a elektrotechniky.

Učebné pomôcky sú nevyhnutnou, no často zanedbávanou súčasťou výuky. Študenti si vďaka nim môžu lepšie predstaviť a pochopiť problematiku daného učiva. Kombinujú tak príjemne s užitočným, kedy sa študent môže lepšie zoznámiť s hardwarom, softwarom a celkovým fungovaním učebnej pomôcky.
Avšak, takéto pomôcky bývajú častokrát príliš zložité a drahé \cite{Hor}. Z toho dôvodu, je ich použitie pri výučbe nepraktické.

Za cieľom sprístupnenia experimentálnych modulov širokej verejnosti prišli na ústave Automatizácie, merania a aplikovanej informatiky
Strojníckej fakulty Slovenskej technickej univerzity v Bratislave s projektom AutomationShield, ktorý ponúka pomerne jednoduché a cenovo dostupné experimentálne moduly ako Open-source\footnote[1]{Open-source je zo všeobecného pohľadu akákoľvek informácia ktorá je dostupná verejnosti bez poplatku(s voľným prístupom), s ohľadom na fakt, že jej voľné šírenie zostane zachované.} študentské projekty.

Vhodnou platformou na implementáciu týchto modulov sú napríklad prototypizačné dosky Arduino ktoré sú taktiež open-source. Ich nízka cena a celosvetová popularita, spojená s obrovským množstvom návodov, informácii a pomôcok, vytvára ideálnu platformu pre začínajúcich, ako aj pokročilých, programátorov, elektrotechnikov alebo hobby nadšencov.

V bakalárskej práci je opísaný postup výroby a fungovania shieldu s dôrazom na zrozumiteľnosť jednotlivých aspektov aj čitateľom, ktorý o danej téme nie sú dokonale oboznámený. Na začiatku bakalárskej práce, v hardwarovej časti, je opísaný základný princíp fungovania shieldu a následne jeho jednotlivé súčiastky. Pochopenie fungovania jednotlivých súčiastok shieldu je kritické pre správnu manipuláciu užívateľa s jeho jednotlivými časťami. Poslednú časť tvorí tvorba dosky plošných spojov pre shield v programe DipTrace.

V softwarovej časti sú bližšie predstavené jednotlivé charakteristické funkcie shieldu. Funkcie sú usporiadané do logických celkov pre ľahšiu prácu užívateľa s kódom.

