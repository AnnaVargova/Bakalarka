\chapter*{Úvod}
\label{UVOD}
\addcontentsline{toc}{chapter}{Úvod}

%V úvode autor podrobnejšie ako v predhovore, pritom výstižne a krátko charakterizuje stav poznania alebo praxe v špecifickej oblasti, %ktorá je predmetom záverečnej práce. Autor presnejšie ako v predhovore vysvetlí ciele práce, jej zameranie, použité metódy a stručne %objasní vzťah práce k iným prácam podobného zamerania. V úvode netreba zachádzať hlbšie do teórie. Netreba podrobne opisovať metódy, %experimentálne výsledky, ani opakovať závery prípadne odporúčania.
%Úvod začína na novej  strane.


Cielom tejto bakalárskej práce je návrch, výroba a naprogramovanie modernej učebnej pomôcky ktorá slúži na výuku základov teórie riadenia a elektrotechniky.

Učebné pomôcky sú nevyhnutnou, no často zanedbávanou súčasťou výuky. študenti si vďaka nim môžu lepšie predstaviť a pochopiť problematiku daného učiva. Kombinujú tak príjemne s užitočným, kedy sa Študent môže lepšie zoznámiť s hardwareom a systémom fungovania učebnej pomôcky.
Avšak, takéto pomôcky bývajú častokrát príliž sofistikované a drahé a z toho dôvodu, je ich použitie pri výučbe nepraktické. 

Za cielom sprístupnenia experimentálnych modulov širokej verejnosti prišli na ústave Automatizácie, merania a aplikovanej informatiky
Strojníckej fakulty Slovenskej technickej univerzity v Bratislave s projektom AutomationShield, ktorý ponúka pomerne jednoduché a cenovo dostupné experimentálne moduly ako Open-source študentské projekty.  