\section{Software}

Programovacie rozhranie pre platformy arduino sa nazýva Arduino IDE\footnote[5]{Arduino Integrated Development Environment.} a využíva programovací jazyk C++ resp. jeho podobu, s pridanými špecializovanými príkazmi a funkciami priamo pre arduino IDE. Príkazy sú na prvý pohľad zrozumiteľnejšie ako ich skompilovaná\footnote[6]{Kompilácia je preklad zdrojového kódu do podoby ktorú vie procesor prečítať a spracovať.} podoba v jazyku C++, no funkcie resp. schopnosti príkazu sú rovnaké. Preto je arduino vhodným prostriedkom na programovanie ako pre začiatočníkov, tak aj pre skúsenejších programátorov. 

Pri tvorbe programovej časti AeroShieldu je dôležité uvedomiť si fakt že doska vzniká v rámci projektu AutomationShield. Tým že je tento projekt opensource, ktokoľvek môže kód upravovať a vylepšovať, je preto dôležité aby funkcie navádzali používateľov na ich správne použitie a aby boli čo najviac prehľadné. Z tohoto dôvodu bola vytvorená knižnica AutomationShield ktorá v sebe zahŕňa najviac používané funkcie. Predstavme si situáciu kedy v programe ktorý píšeme potrebujeme premenu jednotiek z metrov na centimetre. Pokiaľ takúto funkciu potrebujeme použiť v kóde jeden krát, môžeme túto funkciu napísať priamo do kódu. Avšak pokiaľ túto funkciu využívame častejšie, dáva zmysel uložiť ju mimo kód a následne túto funkciu zavolať naspäť v prípade jej potreby. Sprehľadňuje sa tak vzniknutý kód a znižuje sa možnosť chýb vďaka monotónnym kopírovaniam tej istej funkcie. 

Takúto možnosť externých preddefinovaných funkcií prístupných na zavolanie ponúka objektovo orientované programovanie(OOP) v jazyku C++. Zvyčajne sa vytvárajú dva súbory resp. knižnice, z ktorých jedna sa nazýva "header" alebo hlavička s koncovkou .h a druhá, "source" alebo zdrojový dokument s koncovkou .cpp. Header slúži ako akýsi navádzač a sklad pre premenné a funkcie, ktorý následne komunikuje so source dokumentom v ktorom sú uložené samotné funkcie. 

\subsection{Header}

Header súbor má niekoľko náležitostí ktoré obsahuje. Vytvárame v ňom "class" alebo triedu ktorá v sebe zahŕňa funkcie a premenné ktoré sa nazývajú "objects" alebo objekty. Class teda obsahuje podmnožinu objectov ktoré vieme prepájať a spájať vo väčšie celky, vďaka čomu vieme dosiahnuť veľmi komplexné funkcie. Tieto funkcie a premenné môžu byť buď "public" teda verejné a prístupné aj mimo súbor alebo "privat" teda súkromné ktoré su prístupné len v knižniciach header a source. Deklarácia takejto triedy vyzerá nasledovne: 


\begin{lstlisting}
	class AeroShield{		// Deklaracia triedy
		public :		// Verejna cast
		void FirstObject();	// Deklaracia funkcie
		
		private :		// Sukromna cast
		float FirstVariable;	// Deklaracia premennej
	};				// Koniec triedy
\end{lstlisting}
\newpage

V našom prípade nám postačuje jedna trieda ktorá sa nazýva AeroShield a má v sebe jednu funkciu s názvom FirstObject() v časti public a jednu premennú FirstVariable typu float v časti pivate. Rozdelenie na public a privat má zmysel hlavne v prípade ak chceme mať zadefinované isté premenné, pri ktorých nechceme aby sa dala externe zmeniť ich hodnota alebo typ. V prípade privat, takáto zmena nie je možná, jediná možnosť ako premennú zmeniť, je jej ručné prepísanie v súbore. V časti private deklarujeme funkcie ktoré následne využívame v rámci triedy a slúžia ako pomocné funkcie pri tvorbe komplexnejších častí kódu. V časti public sú funkcie viditeľné a schopné interagovať s inými triedami ako aj s inými knižnicami. 


